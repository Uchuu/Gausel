%=====================================================================%
%                        Making an report                             %
%=====================================================================%
\documentclass[11pt,a4paper,oneside]{article}

%=====================================================================%
%                        Packages import                              %
%=====================================================================%
\usepackage{graphicx,amssymb,amstext,amsmath,amsthm,hyperref,indentfirst}
\usepackage{array,amsfonts}
\usepackage{xspace}
\usepackage[utf8]{inputenc}
\usepackage[T1]{fontenc}
\usepackage{ulem}
\usepackage{caption}

%=====================================================================%
%                 Options to make stuff look pretty                   %
%=====================================================================%
\hypersetup{
  pdfborder={0 0 0},
  colorlinks=true,
  linkcolor=blue,
  citecolor=blue,
  urlcolor=blue,
  pdftitle={
    Solving A Simple Stupid Linear System
  }
}

% Used to specify the space between the lines of the matrices and tabulars.
\newcommand\matandtabspace{
.4em
}
% Original system as a tabular.
\newcommand\originaltab{
\begin{tabular}{ c  c  c  c  c  c  c }
   &  & $(a12 * x_{2})$ & $+$ & $(a13 * x_{3})$ & $=$ & $b1$\\[\matandtabspace]
  $(a21 * x_{1})$ & $+$ & $(a22 * x_{2})$ & &  & $=$ & $b2$\\[\matandtabspace]
   &  & $(a32 * x_{2})$ & $+$ & $(a33 * x_{3})$ & $=$ & $b3$\\[\matandtabspace]
\end{tabular}
}
% Original matrix.
\newcommand\originalmat{
\begin{pmatrix}
  0 & a12 & a13\\[\matandtabspace]
  a21 & a22 & 0\\[\matandtabspace]
  0 & a32 & a33\\[\matandtabspace]
\end{pmatrix}
}
% Original vector.
\newcommand\originalvec{
\begin{pmatrix}
  b1\\[\matandtabspace]
  b2\\[\matandtabspace]
  b3\\[\matandtabspace]
\end{pmatrix}
}
% Triangularized system as a tabular.
\newcommand\triangletab{
\begin{tabular}{ c  c  c  c  c  c  c }
  $(a21 * x_{1})$ & $+$ & $(a22 * x_{2})$ & &  & $=$ & $b2$\\[\matandtabspace]
   &  & $(a32 * x_{2})$ & $+$ & $(a33 * x_{3})$ & $=$ & $b3$\\[\matandtabspace]
   &  &  &  & $\frac{(a13 * x_{3})}{a32}$ & $=$ & $(((a32 * b1) - (a12 * b3)) / a32)$\\[\matandtabspace]
\end{tabular}
}
% Triangularized matrix.
\newcommand\trianglemat{
\begin{pmatrix}
  a21 & a22 & 0\\[\matandtabspace]
  0 & a32 & a33\\[\matandtabspace]
  0 & 0 & \frac{((a32 * a13) - (a12 * a33))}{a32}\\[\matandtabspace]
\end{pmatrix}
}
% Triangularized vector.
\newcommand\trianglevec{
\begin{pmatrix}
  b2\\[\matandtabspace]
  b3\\[\matandtabspace]
  \frac{((a32 * b1) - (a12 * b3))}{a32}\\[\matandtabspace]
\end{pmatrix}
}
% Variable vector.
\newcommand\variablevec{
\begin{pmatrix}
  x_1\\[\matandtabspace]
  x_2\\[\matandtabspace]
  x_3\\[\matandtabspace]
\end{pmatrix}
}
% Solution as a tabular.
\newcommand\solutiontab{
\begin{tabular}{ c c l }
  $x_1$ & $=$ & $\frac{((((a32 * a13) * a32) * b2) - (a22 * (((a32 * a13) * b3) - (a33 * (a12 * a32)))))}{(((a32 * a13) * a32) * a21)}$\\[\matandtabspace]
  $x_2$ & $=$ & $\frac{(((a32 * a13) * b3) - (a33 * (a12 * a32)))}{((a32 * a13) * a32)}$\\[\matandtabspace]
  $x_3$ & $=$ & $\frac{(a12 * a32)}{(a32 * a13)}$\\[\matandtabspace]
\end{tabular}
}
% Solution as a vector.
\newcommand\solutionmat{
\begin{pmatrix}
  \frac{((((a32 * a13) * a32) * b2) - (a22 * (((a32 * a13) * b3) - (a33 * (a12 * a32)))))}{(((a32 * a13) * a32) * a21)}\\[\matandtabspace]
  \frac{(((a32 * a13) * b3) - (a33 * (a12 * a32)))}{((a32 * a13) * a32)}\\[\matandtabspace]
  \frac{(a12 * a32)}{(a32 * a13)}\\[\matandtabspace]
\end{pmatrix}
}

%==================================================================%
%                         Title and author(s)                      %
%==================================================================%
\title{
  Solving A Simple Stupid Linear System
}
\author{
}

%==================================================================%
%                    Let's start the document                      %
%==================================================================%
\begin{document}
\normalem

%==================================================================%
%                              Title                               %
%==================================================================%
\maketitle

%==================================================================%
%                         Document body                            %
%==================================================================%

\section{Explanation}
The matrices, vectors, and systems are available to use anywhere in the tex file
using several commands. Sections~\ref{original}~\ref{triangle}
and~\ref{solution} show how to use them in context.
Please note that matrices and vectors have to be put in a math environment,
but tabulars do not.\
Here are the commands defined:
\begin{itemize}
\item \verb+\originaltab+:\\ \originaltab
\item \verb+\originalmat+:\\ $\originalmat$
\item \verb+\originalvec+:\\ $\originalvec$
\item \verb+\triangletab+:\\ \triangletab
\item \verb+\trianglemat+:\\ $\trianglemat$
\item \verb+\trianglevec+:\\ $\trianglevec$
\item \verb+\variablevec+:\\ $\variablevec$
\item \verb+\solutiontab+:\\ \solutiontab
\item \verb+\solutionmat+:\\ $\solutionmat$
\end{itemize}
\section{Original system}
\label{original}
Assuming the unknown variable vector is represented as $\variablevec$,
what we do is solve the following problem:
\[
\originalmat
 *
\variablevec
=
\originalvec
. \]
Equivalently, we can consider the following linear system:\\
\originaltab

\section{Triangularized system}
\label{triangle}
Now, using Gaussian Elimination we obtain the following matrix equation:
\[
\trianglemat
*
\variablevec
=
\trianglevec
\]
Equivalently, the linear system:\\
\triangletab

\section{Solution}
\label{solution}
Now, the solution for the system described in Section~\ref{original} and
triangularized in Section~\ref{triangle} is:
\[ \solutionmat \]
It can also be written as equalities:\\
\solutiontab

\section{Conclusion}
I don't know who did this but it's pretty fucking awesome.
\newline

See ya.

%=================================================================%
%                      End of the Document                        %
%=================================================================%
\end{document}
